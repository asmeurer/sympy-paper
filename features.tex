% Features to discuss in-depth:

SymPy has an extensive feature set that encompasses too much to cover
in-depth here. Bedrock areas, such a Calculus, receive their own sub-sections
below. Additionally, Table \ref{features-table} describes other capabilities
present in the SymPy code base. This gives a sampling from the breadth of
topics and application domains that SymPy services.

\begin{table}
\label{features-table}
\caption{SymPy Features and Descriptions}
\begin{tabular}[htbc]{|l|p{0.7\linewidth}|}
\hline
\textbf{Feature} & \textbf{Description} \\
\hline
Discrete Math & Summations, products, binomial coefficients,
    prime number tools, integer factorization, Diophantine equation solving, and
    boolean logic representation, equivalence testing, and inference.\\
Concrete Math & Tools for determining whether summation and product
    expressions are convergent, absolutely convergent, hypergeometric, and
    other properties. May also compute Gosper's normal form
    \cite{petkovvsek1996bak} for two univariate polynomials.\\
Plotting & Hooks for visualizing expressions via matplotlib \cite{Hunter:2007}
    or as text drawings when lacking a graphical back-end.\\
Geometry & Allows the creation of 2D geometrical entities,
    such as lines and circles. Enables queries on these entities, including
    asking the area of an ellipse, checking for collinearity of a set of
    points, or finding the intersection between two lines.\\
Statistics & Support for a random variable type as well as the ability to
    declare this variable from prebuilt distribution functions such as
    Normal, Exponential, Coin, Die, and other custom distributions.\\
Polynomials & Computes polynomial algebras over various coefficient domains
    ranging from the simple (e.g. polynomial division) to the advanced
    (e.g. Gr\"obner bases \cite{adams1994introduction} and multivariate
    factorization over algebraic number domains).\\
Sets & Representations of empty, finite, and infinite sets. This includes
    special sets such as for all natural, integer, and complex numbers.\\
Series & Implements series expansion, sequences, and limit of sequences.
    This includes special series, such as Fourier and power series.\\
Vectors & Provides basic vector math and differential calculus with respect
    to 3D Cartesian coordinate systems.\\
Matrices & Tools for creating matrices of symbols and expressions.
    This is capable of both sparse and dense representations and performing
    symbolic linear algebraic operations (e.g. inversion and factorization).\\
Combinatorics \& Group Theory & Implements permutations, combinations,
    partitions, subsets,
    various permutation groups (such as polyhedral, Rubik, symmetric,
    and others), Gray codes \cite{Nijenhuis1978combinatorial},
    and Prufer sequences \cite{biggs1976graph}.\\
Code Generation & Enables generation of compilable and executable
    code in a variety of different programming languages directly from
    expressions. Target languages include C, Fortran, Julia, JavaScript,
    Mathematica, Matlab and Octave, Python, and Theano.\\
Tensors & Symbolic manipulation of indexed objects.\\
Lie Algebras & Represents Lie algebras and root systems.\\
Cryptography & Represents block and stream ciphers, including
    shift, Affine, substitution, Vigenere's, Hill's, bifid, RSA, Kid RSA,
    linear-feedback shift registers, and Elgamal encryption\\
Special Functions & Implements a number of well known special functions,
    including Dirac delta, Gamma, Beta, Gauss error functions, Fresnel
    integrals, Exponential integrals, Logarithmic integrals, Trigonometric
    integrals, Bessel, Hankel, Airy, B-spline, Riemann Zeta, Dirichlet eta,
    polylogarithm, Lerch transcendent, hypergeometric, elliptic integrals,
    Mathieu, Jacobi polynomials, Gegenbauer polynomial, Chebyshev polynomial,
    Legendre polynomial, Hermite polynomial, Laguerre polynomial, and
    spherical harmonic functions.\\
\hline
\end{tabular}
\end{table}


% Basic operations (the core)
\subsection{Basic Operations}

\input{features_basic_operations}

\subsection{Calculus}

% Solvers (regular equations, maybe also mention other types of solvers like ODEs/recurrence/Diophantine)
\subsection{Solvers}
%% Solvers in SymPy


SymPy has a solvers module to solve equations symbolically. There are two
submodules to solve algebraic equations in SymPy, referred to as old solve
i.e. \texttt{solve} and new solve i.e. \texttt{solveset}. Solveset is
introduced with several design changes with respect to old \texttt{solve} to
resolve the issues with old \texttt{solve}, for example old \texttt{solve}'s
input API has many flags which are not needed and they make it hard for the
user and the developers to work on solvers. In contrast to old solve, the
\texttt{solveset} has a clean input API, It only asks for the much needed
information from the user, following are the function signatures of old and new
solve:

\begin{verbatim}
solve(f, *symbols, **flags)  # old solve
solveset(f, symbol, domain)  # new solve
\end{verbatim}

The old \texttt{solve} has an inconsistent output API for various types of
inputs, whereas the \texttt{solveset} has a canonical output API which is
achieved using sets. It can consistently return various types of solutions.

\begin{itemize}
\item Single solution
\end{itemize}
\begin{verbatim}
>>> solveset(x - 1)
>>> {1}
\end{verbatim}

\begin{itemize}
\item Finite set of solution, quadratic equation
\end{itemize}
\begin{verbatim}
>>> solveset(x**2 - pi**2, x)
{-pi, pi}
\end{verbatim}

\begin{itemize}
\item No Solution
\end{itemize}
\begin{verbatim}
>>> solveset(1, x)
EmptySet()
\end{verbatim}

\begin{itemize}
\item Interval of solution
\end{itemize}
\begin{verbatim}
>>> solveset(x**2 - 3 > 0, x, domain=S.Reals)
(-oo, -sqrt(3)) U (sqrt(3), oo)
\end{verbatim}

\begin{itemize}
\item Infinitely many solutions
\end{itemize}
\begin{verbatim}
>>> solveset(sin(x) - 1, x, domain=S.Reals)
ImageSet(Lambda(_n, 2*_n*pi + pi/2), Integers())
>>> solveset(x - x, x, domain=S.Reals)
(-oo, oo)
>>> solveset(x - x, x, domain=S.Complexes)
S.Complexes
\end{verbatim}

\begin{itemize}
\item Linear system: finite and infinite solution for determined, under
determined and over determined problems.
\end{itemize}
\begin{verbatim}
>>> A = Matrix([[1, 2, 3], [4, 5, 6], [7, 8, 10]])
>>> b = Matrix([3, 6, 9])
>>> linsolve((A, b), x, y, z)
{(−1,2,0)}
>>> linsolve(Matrix(([1, 1, 1, 1], [1, 1, 2, 3])), (x, y, z))
{(-y - 1, y, 2)}
\end{verbatim}

The new solve i.e. \textbf{solveset} is under active development and is a
planned replacement for \textbf{solve}, Hence there are some features which are
implemented in solve and is not yet implemented in solveset. The table below
show the current state of old and new solve.

\hfill

\begin{tabular}{ |p{4cm}|p{3cm}|p{3cm}|  }
\hline
\multicolumn{3}{|c|}{Solveset vs Solve} \\
\hline
Feature& solve &solveset \\
\hline
Consistent Output API & No & Yes \\
Consistent Input API & No & Yes \\
Univariate & Yes & Yes\\
Linear System& Yes & Yes (linsolve) \\
Non Linear System& Yes & Not yet \\
Transcendental& Yes & Not yet \\
\hline
\end{tabular}

\hfill \break

Below are some of the examples of old \textbf{solve} :

\begin{itemize}
\item Non Linear (multivariate) System of Equation: Intersection of a circle
and a parabola.
\end{itemize}
\begin{verbatim}
>>> solve([x**2 + y**2 - 16, 4*x - y**2 + 6], x, y)
[(-2 + sqrt(14), -sqrt(-2 + 4*sqrt(14))),
 (-2 + sqrt(14), sqrt(-2 + 4*sqrt(14))),
 (-sqrt(14) - 2, -I*sqrt(2 + 4*sqrt(14))),
 (-sqrt(14) - 2, I*sqrt(2 + 4*sqrt(14)))]
\end{verbatim}

\begin{itemize}
\item Transcendental Equation
\end{itemize}
\begin{verbatim}
>>> solve(x + log(x))**2 - 5*(x + log(x)) + 6, x)
[LambertW(exp(2)), LambertW(exp(3))]
>>> solve(x**3 + exp(x))
[-3*LambertW((-1)**(2/3)/3)]
\end{verbatim}


% Matrices (worth including to stress that they are symbolic)
\subsection{Matrices}

% Physics module (some sampling, to show that it is there)
\subsection{Physics}

% Logic module
\subsection{Logic}

SymPy supports construction and manipulation of boolean expressions
through the \texttt{logic} module. SymPy symbols can be used as 
propositional variables and also be substituted as \texttt{True} 
or \texttt{False}. A good number of manipulation features for boolean 
expressions have been implemented in the \texttt{logic} module.

\subsubsection{Constructing boolean expressions}

A boolean variable can be declared as a SymPy symbol. Python
operators \&, \textbar  and \textasciitilde are overloaded for logical \texttt{And}, 
\texttt{Or} and \texttt{negate}. Several others like \texttt{Xor},
\texttt{Implies} can be constructed with \^{}, \textgreater\textgreater respectively.  
The above are just a shorthand, expressions can also be constructed
by directly calling \texttt{And()}, \texttt{Or()}, \texttt{Not()},
\texttt{Xor()}, \texttt{Nand()}, \texttt{Nor()}, etc.
The boolean symbols can also be substituted \texttt{True} or \texttt{False}

\begin{verbatim}
>>> e = (x & y) | z
>>> e.subs({x: True, y: True, z: False})
True
\end{verbatim}

\subsubsection{CNF and DNF}

Any boolean expression can be converted to conjunctive normal 
form, disjunctive normal form and negation normal form. The 
API also permits to check if a boolean expression is in any 
of the above mentioned forms.

\begin{verbatim}
>>> to_cnf((A & B) | C)
And(Or(A, C), Or(B, C))
>>> to_dnf(A & (B | C))
Or(And(A, B), And(A, C))
>>> is_cnf((x | y) & z)
True
>>> is_dnf((x & y) | z) 
True
\end{verbatim}

\subsubsection{Simplification and Equivalence}

The module supports simplification of given boolean expression
by making deductions on it. Equivalence of two expressions can
also be checked. If so, it is possible to return the mapping of 
variables of two expressions so as to represent the 
same logical behaviour.

\begin{verbatim}
>>> e = a & (~a | ~b) & (a | c)
>>> simplify(e)
And(Not(b), a)
>>> e1 = a & (b | c)
>>> e2 = (x & y) | (x & z)
>>> bool_map(e1, e2)
(And(Or(b, c), a), {b: y, a: x, c: z})
\end{verbatim}

\subsubsection{SAT solving}

The module also supports satisfiability checking of a given
boolean expression. If satisfiable, it is possible to return 
a model for which the expression is satisfiable. The API also
supports returning all possible models. The SAT solver has 
a clause learning DPLL algorithm implemented with watch 
literal scheme and VSIDS heuristic\cite{moskewicz2008method}.

\begin{verbatim}
>>> satisfiable(a & (~a | b) & (~b | c) & ~c)
False
>>> satisfiable(a & (~a | b) & (~b | c) & c)
{b: True, a: True, c: True}
\end{verbatim}


% Series module (Formal Power Series, Fourier Series)
\subsection{Series}
% Series expansion (Differentiate between the two approaches being used)
\subsubsection{Series Expansion}

SymPy is able to calculate the symbolic series expansion of an arbitrary series
or expression involving elementary and special functions and multiple
variables. For this it has two different implementations.

The first approach stores a series as a core object. Each function has its
specific implementation of its expansion. It is able to evaluate the Puiseux
series expansion about any point.

\begin{verbatim}
>>> from sympy import *
>>> x, y = symbols('x, y')
>>> series(sin(x+y) + cos(x*y), x, 0, 2)
1 + sin(y) + x*cos(y) + O(x**2)
\end{verbatim}

The newer and much faster approach called Ring Series makes use of the
observation that a truncated Taylor series, is in fact a polynomial.
Ring Series uses the efficient representation and operations of sparse
polynomials. The choice of sparse polynomials is deliberate as it performs
well in a wider range of cases than a dense representation. Ring Series gives 
the user the freedom to choose the type of coefficients he wants to have in
his series, allowing the use of faster operations on certain types.

For this, several low level methods for expansion of elementary functions are
implemented using semi-numerical algorithms. All these support Puiseux series
expansion.

\begin{verbatim}
>>> from sympy.polys.ring_series import rs_sin
>>> R, x = ring('x', QQ)
>>> rs_sin(x**2 + x, x, 5)
-1/2*x**4 - 1/6*x**3 + x**2 + x
\end{verbatim}

The function \texttt{rs\_series} makes use of these elementary functions to
intelligently expand an arbitrary SymPy expression. Currently it only supports
expansion about 0 and is under active development. Typical speedup over
the older SymPy implementation is in the range of 20-100x with larger
speedup observed with larger series.

\begin{verbatim}
>>> from sympy.polys.ring_series import rs_series
>>> rs_series(sin(a + b), a, 4)
-1/2*(sin(b))*a**2 + (sin(b)) - 1/6*(cos(b))*a**3 + (cos(b))*a
\end{verbatim}

\subsubsection{Formal Power Series}

SymPy can be used for computing the Formal Power Series of a function.
The implementation is based on the algorithm described in the paper on Formal Power Series\cite{Gruntz93formalpower}.
The advantage of this approach is that an explicit formula for the coefficients
of the series expansion is generated rather than just computing a few terms.

\begin{verbatim}
>>> f = fps(sin(x), x, x0=0)
>>> f.truncate(6)
x - x**3/6 + x**5/120 + O(x**6)
>>> f[15]
-x**15/1307674368000
\end{verbatim}

\subsubsection{Fourier Series}

SymPy provides functionality to compute Fourier Series of a function using
the \texttt{fourier\_series} function. Under the hood it just computes $a0$, $an$, $bn$ using
standard integration formulas.

\begin{verbatim}
>>> L = symbols('L')
>>> f = fourier_series(2 * (Heaviside(x/L) - Heaviside(x/L - 1)) - 1, (x, 0, 2*L))
>>> f.truncate(3)
4*sin(pi*x/L)/pi + 4*sin(3*pi*x/L)/(3*pi) + 4*sin(5*pi*x/L)/(5*pi)
\end{verbatim}


% Features to list, but not discuss in-depth:

% discrete math, concrete math, plotting, geometry, statistics, polys,
% combinatorics/group theory, code generation, tensors, lie algebras,
% cryptography, category theory, special functions, sets, matrix expressions,
% series, or vectors.
\subsection{Other features}
