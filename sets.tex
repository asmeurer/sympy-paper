%% Sets

SymPy supports representation of a wide variety of set, this is achieved by
first defining abstract representation for a smaller number of atomic set
classes and then combining and transforming them using various set operations.

Each of the set class inherits from the base Set class and defines
rules to check membership of a SymPy object, to calculate union, intersection,
set different. In cases we are not able to evaluate these
operations to atomic set classes they are represented as abstract unevaluated
objects.


\begin{itemize}

    % Description of Empty Set sounds too obvious, not sure if we need to keep
    % it.
    \item \textbf{Empty Set}: Nothing is a member of Empty Set. Union with
another set returns the other set and intersection leads to an Empty Set.

    \item \textbf{Universal Set}: Everything is a member of Universal Set.
Union of Universal Set with any set gives Universal Set and intersection leads
to the other set itself.

    \item \textbf{Finite Set} is functionally equivalent to python's set
object. Its members can be any object including strings and other sets
themselves.

    \item \textfb{Range} implements a range of integers and is defined by
specifying a start value, an end value and a step size. Range is functionally
equivalent to python's range except the fact that it accepts infinity at end
points allowing us to represent infinite ranges.


    \item \textbf{Real Interval} is specified by giving the start and end point
and specifying if it is open or closed in these respective ends. The set of
real numbers is represented as a special case of a real interval where the
start point is negative infinite and the end point is positive infinite.

\end{itemize}


%% Operations

Other than unevaluated classes of of Union, Intersection and Set Difference
operations, we have following set classes.

\begin{itemize}

    \item Product Set abstractly defines the Cartesian product of two or more
sets. Product Set is useful when representing higher dimensional spaces. For
example to represent a three dimensional space we simple take the Cartesian
product of three Real sets.

    \item Image Set represents the set of image a function when applied to a
particular set. In notation Image Set of a function F w.r.t a set S is \{ F(x)
| x \in S \} In particular we use Image Set to represent the set of infinite
solutions from trigonometric equations.


    \item Condition Set represents subset of a set who's members which
satisfies a particular condition. In notation Condition Set of set S w.r.t to a
condition H is \{x | H(x), x \in S \}. We use Condition Set to represent the
set of solutions of an equation or an inequality where the equation or the
inequality is the condition and the set is the domain in which we aim to find
the solution.


\end{itemize}





%% Explaining it later because it is a special case of Image Set rather being
%% something atomic

    \textbf{Complex Region}

%% Representations achievable through application of Operations on atomic set
%% types mentioned above.


%% Special Cases
